\documentclass{beamer}

\usepackage{comment}
\usepackage{color}

\title{An Overview of the PFLOTRAN Input Deck}
\author{Glenn Hammond}
\date{\today}

\begin{document}

\frame{\titlepage}

\begin{comment}
\begin{frame}
%Freaking manual table of contents since beamer is stupid!!!!
\footnotesize
\begin{itemize}
\item[] {\color{blue}Syntax}
\item[] {\color{blue}Control Cards}
\begin{itemize}
\item[] SIMULATION
\end{itemize}
\item[] {\color{blue}Required Cards}
\begin{itemize}
\item[] GRID
\item[] REGION
\item[] MATERIAL\_PROPERTY
\item[] STRATA
\item[] TIME
\item[] INITIAL\_CONDITION
\end{itemize}
\item[] {\color{blue}Required Cards for Flow}
\begin{itemize}
\item[] FLOW\_CONDITION
\end{itemize}
\item[] {\color{blue}Required Cards for Reactive Transport}
\begin{itemize}
\item[] CHEMISTRY
\item[] TRANSPORT\_CONDITION
\item[] CONSTRAINT
\end{itemize}
\item[] {\color{blue}Optional Cards}
\begin{itemize}
\item[] OUTPUT
\item[] FLUID\_PROPERTY
\end{itemize}
\end{itemize}
\end{frame}
\end{comment}

\section{Syntax}
\section{Syntax}

\begin{frame}[containsverbatim,allowframebreaks]\frametitle{Syntax}

The PFLOTRAN input deck is defined using keywords or cards that are associated with data.  The file is designed with modularity in mind and the cards need not be in any particular order.  Many cards open a section that provides additional cards and is terminated by an ``END'' or ``/''.  For example:

\begin{semiverbatim}
CARD1
  CARD2 value value
  CARD3
    CARD4 value
    CARD5 value
  /
END
\end{semiverbatim}


Several options exist for commenting out lines or sections of the input file:
\begin{itemize}
\item Individual lines may be commented out by placing a colon (:) or exclamation point (!) at the beginning of line (i.e. before any text).
\item The SKIP/NOSKIP cards may be used to comment out a large number of lines.  SKIP/NOSKIP may be used in a nested manner.  But beware!  One must ensure consistency (i.e. \# SKIPs == \# NOSKIPs).
\item Exclamation points (!) are often used to place comments at the end of a line, but serve no purpose and are solely for markup (i.e. PFLOTRAN will continue to read the string if it is expecting additional entry data).
\end{itemize}

\begin{semiverbatim}
CARD1
  skip
  CARD2 value value
  CARD3
    skip
    CARD4 value
    CARD5 value
    noskip
  /
  noskip
END
\end{semiverbatim}

\end{frame}


\section{File Layout}
\subsection{Layout}

\begin{frame}[fragile,containsverbatim]\frametitle{Layout}

\begin{semiverbatim}
  SIMULATION
    # declares the process models used.
  END

  SUBSURFACE
    # defines the flow and transport process models
  END
  
  PM_BLOCK(s)
    # defines other process models
  END

\end{semiverbatim}

\end{frame}

\begin{frame}[fragile,containsverbatim]\frametitle{SUBSURFACE Layout (Minimal)}

\begin{semiverbatim}
  SUBSURFACE
    GRID               # discretization
    MATERIAL_PROPERTY  # material properties
    OUTPUT             # screen/file output
    TIME               # final time, time stepping, etc
    REGION             # regions in domain
    FLOW_CONDITION     # state variables
    INITIAL_CONDITION  # couple FLOW_CONDITION and REGION
    BOUNDARY_CONDITION # couple FLOW_CONDITION and REGION
    STRATA             # couple MATERIAL_PROPERTY and
  END                  #   REGION
\end{semiverbatim}

\end{frame}

\begin{frame}[fragile,containsverbatim]\frametitle{Coupling REGIONs to Properties and States}

\begin{semiverbatim}
  INITIAL_CONDITION
    FLOW_CONDITION pressure
    REGION source_zone
  END
  
  BOUNDARY_CONDITION inlet
    FLOW_CONDITION injection
    REGION west_face
  END
  
  STRATA
    MATERIAL soil
    REGION source_zone
  END
\end{semiverbatim}

\end{frame}

%-----------------------------------------------------------------------------
\begin{frame}[fragile,containsverbatim]\frametitle{}
\small
\begin{semiverbatim}
PFLOTRAN_DIR/shortcourse/exercises/1D_variably_saturated_flow
\end{semiverbatim}
\end{frame}

%-----------------------------------------------------------------------------
\section{Control Cards}

\subsection{SIMULATION}

\begin{frame}[fragile,containsverbatim]\frametitle{SIMULATION}

\begin{itemize}
\item[] \textbf{Purpose:} Defines
\begin{itemize}
  \item Type of simulation
  \item Process models required
  \item Process model specific options
  \item Checkpoint/restart
\end{itemize}
\begin{comment}
\item[] \textbf{Example uses:}
\begin{itemize}
  \item
\end{itemize}
\end{comment}
\item[] \textbf{Key sub-cards:}
\begin{itemize}
\item[] \verb|SIMULATION_TYPE|
\item[] \verb|  SUBSURFACE|
\item[] \verb|  GEOMECHANICS_SUBSURFACE|
\item[] \verb|  ...|
\item[] \verb|PROCESS_MODELS|
\item[] \verb|  SUBSURFACE_FLOW|
\item[] \verb|  SUBSURFACE_TRANSPORT|
\item[] \verb|  UFD_DECAY|
\item[] \verb|  ...|
\item[] \verb|CHECKPOINT|
\item[] \verb|RESTART|
\end{itemize}
\end{itemize}

\end{frame}

%-----------------------------------------------------------------------------
\begin{frame}[fragile]\frametitle{SIMULATION Example: Flow and Transport}

\begin{semiverbatim}
SIMULATION
  SIMULATION_TYPE SUBSURFACE
  PROCESS_MODELS
    SUBSURFACE_FLOW flow
      MODE RICHARDS
    /
    SUBSURFACE_TRANSPORT transport
      MODE GIRT
    /
  /
END
\end{semiverbatim}

\end{frame}

%-----------------------------------------------------------------------------
\begin{frame}[fragile]\frametitle{SIMULATION Example: Multiphase Flow}

\begin{semiverbatim}
SIMULATION
  SIMULATION_TYPE SUBSURFACE
  PROCESS_MODELS
    SUBSURFACE_FLOW flow
      MODE GENERAL
      OPTIONS
        PHASE_CHANGE_EPSILON
        ARITHMETIC_GAS_DIFFUSIVE_DENSITY
      /
    /
  /
END
\end{semiverbatim}

\end{frame}

%-----------------------------------------------------------------------------
\begin{frame}[fragile]\frametitle{SIMULATION Example: With Checkpointing}
\begin{semiverbatim}
SIMULATION
  SIMULATION_TYPE SUBSURFACE
  PROCESS_MODELS
    SUBSURFACE_FLOW flow
      MODE TH
    /
  /
  CHECKPOINT
    PERIODIC TIMESTEP 10
    PERIODIC TIME y 100.
    TIMES y 137.
    FORMAT HDF5
  /
END
\end{semiverbatim}

\end{frame}


%-----------------------------------------------------------------------------
\section{Required Cards}

\section{GRID Card}

\begin{frame}[fragile,containsverbatim]\frametitle{GRID}

\begin{itemize}
\item[] \textbf{Purpose:} Defines the type of grid/mesh to be employed and parameters describing the grid
\item[] \textbf{Example uses:}
\begin{itemize}
  \item Defining a structured AMR grid
  \item Defining the bounding box and resolution of a structured grid
  \item Defining the origin of a grid
\end{itemize}
\end{itemize}

\end{frame}

\begin{frame}[fragile]\frametitle{GRID: Examples}

\end{frame}

\section{REGION Card}

\begin{frame}[fragile,containsverbatim]\frametitle{REGION}

\begin{itemize}{}
\item[] \textbf{Purpose:} To define a region within the model domain to be associated with another entity within the simulation.
\item[] \textbf{Example uses:}
\begin{itemize}
  \item To assign a recharge boundary condition to the top surface of the model
  \item To assign material properties to a zone in the model
  \item To assign a source/sink term representing an injection well to cells intercepted by the well string
  \item To an initial condition to a zone in the model
\end{itemize}
\end{itemize}

\end{frame}


\begin{frame}[fragile]\frametitle{REGION: Examples}

\scriptsize
\begin{multicols}{2}

\textbf{Volume:}
\begin{semiverbatim}
REGION all
  COORDINATES
    0. 0. 0.
    100. 100. 10.
  /
/
\end{semiverbatim}
\textbf{Surface:}
\begin{semiverbatim}
REGION river
  COORDINATES
    0. 0. 0.
    0. 100. 10.
  /
  FACE west
/
\end{semiverbatim}

\textbf{Surface:}
\begin{semiverbatim}
REGION river
  BLOCK 1 1 1 100 1 10
  FACE west
/
\end{semiverbatim}

\textbf{Point:}
\begin{semiverbatim}
REGION observation_point
  COORDINATE 50. 50. 5.
/
\end{semiverbatim}

\textbf{List of cells and faces:}
\begin{semiverbatim}
REGION FILE river.h5
\end{semiverbatim}


\end{multicols}

\end{frame}

\subsection{MATERIAL\_PROPERTY}

\begin{frame}[fragile,containsverbatim]\frametitle{MATERIAL\_PROPERTY}

\begin{itemize}
\item[] \textbf{Purpose:} Define material properties for simulation.
\item[] \textbf{Example uses:}
\begin{itemize}
  \item Associate ID with material properties
  \item Specify porosity, permeability, rock/soil particle density
  \item Associate constitutive relations (e.g. saturation functions, relative permeability functions)
\end{itemize}
\item[] \textbf{Key sub-cards:}
\begin{itemize}
  \item[] \verb|ID|
  \item[] \verb|POROSITY|
  \item[] \verb|TORTUOSITY|
  \item[] \verb|PERMEABILITY|
  \item[] \verb|ROCK_DENSITY|
  \item[] \verb|SPECIFIC_HEAT|
  \item[] \verb|CHARACTERISTIC_CURVES|
  \item[] \verb|SOIL_COMPRESSIBILITY|
\end{itemize}
\end{itemize}

\end{frame}

%-----------------------------------------------------------------------------
\begin{frame}[fragile]\frametitle{MATERIAL\_PROPERTY Example: Solute Transport}

\begin{semiverbatim}
MATERIAL_PROPERTY soil
  ID 1
  POROSITY 0.25d0
  TORTUOSITY 1.d0
END
\end{semiverbatim}

\end{frame}

%-----------------------------------------------------------------------------
\begin{frame}[fragile]\frametitle{MATERIAL\_PROPERTY Example: Single Phase}

\begin{semiverbatim}
MATERIAL_PROPERTY soil1
  ID 1
  POROSITY 0.25d0
  TORTUOSITY 1.d0
  SOIL_COMPRESSIBILITY_FUNCTION LEIJNSE
  SOIL_COMPRESSIBILITY 1.d-7
  SOIL_REFERENCE_PRESSURE 1.d5
  PERMEABILITY
    PERM_ISO 1.d-12
  /
  CHARACTERISTIC_CURVES sf1
END
\end{semiverbatim}

\end{frame}

%-----------------------------------------------------------------------------
\begin{frame}[fragile]\frametitle{MATERIAL\_PROPERTY Example: MultiPhase}

\begin{semiverbatim}
MATERIAL_PROPERTY sand
  ID 1
  CHARACTERISTIC_CURVES cc1
  POROSITY 0.25
  TORTUOSITY 0.5
  ROCK_DENSITY 2650.d0 kg/m^3
  THERMAL_CONDUCTIVITY_DRY 0.6d0 W/m-C
  THERMAL_CONDUCTIVITY_WET 1.9d0 W/m-C
  HEAT_CAPACITY 830.d0 J/kg-C
  PERMEABILITY
    PERM_X 1.1d-12
    PERM_Y 1.d-12
    PERM_Z 1.d-12
  /
/
\end{semiverbatim}

\end{frame}

\subsection{STRATA}

\begin{frame}[fragile,containsverbatim]\frametitle{STRATA}

\begin{itemize}
\item[] \textbf{Purpose:} Ties material properties to regions of the physical domain
\item[] \textbf{Example uses:}
\begin{itemize}
  \item Define material IDs for grid cells
  \item Define transient material IDs
\end{itemize}
\item[] \textbf{Key sub-cards:}
\begin{itemize}
  \item[] \verb|REGION|
  \item[] \verb|MATERIAL|
  \item[] \verb|START_TIME|
  \item[] \verb|FINAL_TIME|
\end{itemize}
\end{itemize}

\end{frame}

%-----------------------------------------------------------------------------
\begin{frame}[fragile]\frametitle{STRATA Example: Simple}

\begin{semiverbatim}
STRATA
  REGION layer1
  MATERIAL soil1
END
\end{semiverbatim}

\end{frame}

%-----------------------------------------------------------------------------
\begin{frame}[fragile]\frametitle{STRATA Example: Transient}
\footnotesize
\begin{semiverbatim}
STRATA
  REGION rPCS
  MATERIAL PCS_T1
  START_TIME 0 y
  FINAL_TIME 100 y
END

STRATA
  REGION rPCS
  MATERIAL PCS_T2
  START_TIME 100 y
  FINAL_TIME 200 y
END
 
STRATA
  REGION rPCS
  MATERIAL PCS_T3
  START_TIME 200 y
  FINAL_TIME 10000 y
END
\end{semiverbatim}

\end{frame}

\section{TIME Card}

\begin{frame}[fragile,containsverbatim]\frametitle{TIME}

\begin{itemize}
\item[] \textbf{Purpose:} Specifies critical times during the simulation
\item[] \textbf{Example uses:}
\begin{itemize}
  \item Defining the final time
  \item Defining initial time step size
  \item Defining maximum time step size
\end{itemize}
\end{itemize}

\end{frame}

\begin{frame}[fragile]\frametitle{TIME: Examples}

\end{frame}

\section{INITIAL\_CONDITION Card}

\begin{frame}[fragile,containsverbatim]\frametitle{INITIAL\_CONDITION}

\begin{itemize}
\item[] \textbf{Purpose:} Couples a flow and/or transport conditions with a region to create an initial condition.
\item[] \textbf{Example uses:}
\begin{itemize}
  \item To assign a initial condition.
\end{itemize}
\end{itemize}

\end{frame}

\begin{frame}[fragile]\frametitle{INITIAL\_CONDITION: Examples}

\end{frame}


\section{Required Cards for Flow}
\subsection{FLOW\_CONDITION}

\begin{frame}[fragile,containsverbatim]\frametitle{FLOW\_CONDITION}

\begin{itemize}
\item[] \textbf{Purpose:} Defines parameters/conditions to be associated with flow boundary and initial conditions
\item[] \textbf{Example uses:}
\begin{itemize}
  \item Specify a constant pressures on cell faces
  \item Specify a transient fluxes through cell faces
  \item Define a hydrostatic column of water
\end{itemize}
\item[] \textbf{Key sub-cards:}
\begin{itemize}
  \item[] \verb|TYPE|
  \item[] \verb|  DIRICHLET|
  \item[] \verb|  NEUMANN|
  \item[] \verb|  RATE|
  \item[] \verb|DATUM|
  \item[] \verb|GRADIENT|
  \item[] \verb|PRESSURE|
  \item[] \verb|FLUX|
  \item[] \verb|RATE|
\end{itemize}
\end{itemize}

\end{frame}

%-----------------------------------------------------------------------------
\begin{frame}[fragile]\frametitle{FLOW\_CONDITION Example: Undulating River}

\begin{semiverbatim}
FLOW_CONDITION river
  TYPE
    PRESSURE HYDROSTATIC
  /
  INTERPOLATION LINEAR
  DATUM LIST
    TIME_UNITS d
    0.d0 0.d0 0.d0 34.d0
    10.d0 0.d0 0.d0 39.d0
    50.d0 0.d0 0.d0 33.d0
    100.d0 0.d0 0.d0 34.d0
  /
  PRESSURE 101325 ! Pa
END
\end{semiverbatim}

\end{frame}

%-----------------------------------------------------------------------------
\begin{frame}[fragile]\frametitle{FLOW\_CONDITION Example: Transient Rainfall}

\begin{semiverbatim}
FLOW_CONDITION recharge
  TYPE
    FLUX NEUMANN
  /
  FLUX LIST
    TIME_UNITS yr
    DATA_UNITS cm/yr
    0.d0 25.d0
    1.d0 23.d0
    2.d0 27.d0
    3.d0 22.d0
    4.d0 24.d0
    5.d0 29.d0
  /
END
\end{semiverbatim}

\end{frame}

%-----------------------------------------------------------------------------
\begin{frame}[fragile]\frametitle{FLOW\_CONDITION Example: Permeability-Weighted Injection Well}

\begin{semiverbatim}
FLOW_CONDITION injection_well
  TYPE
    RATE SCALED_VOLUMETRIC_RATE NEIGHBOR_PERM
  /
  RATE 1 m^3/hr
END
\end{semiverbatim}

\end{frame}


\section{Required Cards for Reactive Transport}
\section{CHEMISTRY Card}

\begin{frame}[fragile,containsverbatim]\frametitle{CHEMISTRY}

\begin{itemize}
\item[] \textbf{Purpose:} Defines basis species and all chemical reactions based on those species.  Also defines output options for reactive transport.
\item[] \textbf{Example uses:}
\begin{itemize}
  \item Naming chemistry components
  \item Defining geochemistry:
    \begin{itemize}
      \item Defining basis species
      \item Aqueous speciation reactions
      \item Minerals and mineral reactions
      \item Activity calculations
      \item Database name
    \end{itemize}
\end{itemize}
\end{itemize}

\end{frame}

\begin{frame}[fragile]\frametitle{CHEMISTRY: Examples}

\end{frame}

\subsection{TRANSPORT\_CONDITION}

\begin{frame}[fragile,containsverbatim]\frametitle{TRANSPORT\_CONDITION}

\begin{itemize}
\item[] \textbf{Purpose:} Couples constraints with a type of condition for an initial or boundary condition
\item[] \textbf{Example uses:}
\begin{itemize}
  \item Specifying concentrations 
  \item Specifying types of conditions
\end{itemize}
\item[] \textbf{Key sub-cards:}
\begin{itemize}
  \item[] \verb|TYPE|
  \item[] \verb|  DIRICHLET|
  \item[] \verb|  ZERO_GRADIENT|
  \item[] \verb|  DIRICHLET_ZERO_GRADIENT|
  \item[] \verb|CONSTRAINT_LIST|
  \item[] \verb|TIME_UNITS|
\end{itemize}
\end{itemize}

\end{frame}

%-----------------------------------------------------------------------------
\begin{frame}[fragile]\frametitle{TRANSPORT\_CONDITION Example: Steady}

\begin{semiverbatim}
TRANSPORT_CONDITION source_concentration
  TYPE DIRICHLET
  CONSTRAINT_LIST
    0.d0 source_constraint
  /
END
\end{semiverbatim}

\end{frame}

%-----------------------------------------------------------------------------
\begin{frame}[fragile]\frametitle{TRANSPORT\_CONDITION Example: Transient}

\begin{semiverbatim}
TRANSPORT_CONDITION inlet_conc
  TYPE DIRICHLET_ZERO_GRADIENT
  TIME_UNITS y
  CONSTRAINT_LIST
    0.d0 initial_constraint
    1.d0 pulse_constraint
    2.d0 initial_constraint
  /
END
\end{semiverbatim}

\end{frame}

\section{CONSTRAINT Card}

\begin{frame}[fragile,containsverbatim]\frametitle{CONSTRAINT}

\begin{itemize}
\item[] \textbf{Purpose:} Defines a set of concentrations for the primary or basis species (and potentially minerals) to be referenced by a transport condition.
\item[] \textbf{Example uses:}
\begin{itemize}
  \item Defining free-ion or total component concentrations of primary species
  \item Equilibrating species concentrations with a mineral such as Calcite
  \item Defining mineral volume fractions
  \item Setting a species concentration based on charge balance
\end{itemize}
\end{itemize}

\end{frame}

\begin{frame}[fragile]\frametitle{CONSTRAINT: Examples}

\end{frame}


%-----------------------------------------------------------------------------
\section{Optional Cards}
\section{BOUNDARY\_CONDITION Card}

\begin{frame}[fragile,containsverbatim]\frametitle{BOUNDARY\_CONDITION}

\begin{itemize}
\item[] \textbf{Purpose:} Couples a flow and/or transport conditions with a region to create a boundary condition.
\item[] \textbf{Example uses:}
\begin{itemize}
  \item To assign a boundary condition
\end{itemize}
\end{itemize}

\end{frame}

\begin{frame}[fragile]\frametitle{BOUNDARY\_CONDITION: Examples}

\begin{semiverbatim}
\scriptsize
: west boundary condition
BOUNDARY_CONDITION
  FLOW_CONDITION initial
  TRANSPORT_CONDITION west
  REGION West
END

: east boundary condition
BOUNDARY_CONDITION
  FLOW_CONDITION initial
  TRANSPORT_CONDITION initial
  REGION East
END

\end{semiverbatim}


\end{frame}

[suites]
standard = output_tecplot_point
           output_tecplot_block
           output_vtk
           output_hdf5
           output_biosphere
           output_waste_form_wf
#           output_WIPP_pnl
           output_sc_tecplot
           output_coordinates

standard_parallel = output_tecplot_block-np3
                    output_vtk-np3
                    output_hdf5-np3
                    output_face_vel-np4
                    output_aggregate-np2

[default-test-criteria]
# default criteria for all tests, can be overwritten by specific tests
time = 50 percent
generic = 1.0e-12 absolute
concentration = 1.0e-12 absolute
discrete = 0 absolute
rate = 1.0e-12 absolute
volume fraction = 1.0e-12 absolute
pressure = 1.0e-12 absolute
saturation = 1.0e-12 absolute
charge balance = 1.0e-12 absolute

[output_tecplot_point]
skip_check_regression = True
diff_ascii_output = output_tecplot_point-000.tec

[output_tecplot_block]
skip_check_regression = True
diff_ascii_output = output_tecplot_block-000.tec

[output_vtk]
skip_check_regression = True
diff_ascii_output = output_vtk-000.vtk

# currently cannot compare binary hdf5
[output_hdf5]
skip_check_regression = True
#diff_ascii_output = output_hdf5.h5

# not possible in parallel
#[output_tecplot_point-np3]
#diff_ascii_output = output_tecplot_point-np3-000.tec

[output_tecplot_block-np3]
skip_check_regression = True
np = 3
diff_ascii_output = output_tecplot_block-np3-000.tec

# geh: output_tecplot_febrick-np3 currently decomposed incorrectly as 3 cells 
# are on a single process.  But I don't have time to sort this output right now.
#[output_tecplot_febrick-np3]
#skip_check_regression = True
#np = 3
#diff_ascii_output = output_tecplot_febrick-np3-000.tec

[output_vtk-np3]
skip_check_regression = True
np = 3
diff_ascii_output = output_vtk-np3-000.vtk

# currently cannot compare binary hdf5
[output_hdf5-np3]
np = 3
skip_check_regression = True
#diff_ascii_output = output_hdf5-np3.h5

[output_biosphere]
skip_check_regression = True
diff_ascii_output = output_biosphere-0.bio

[output_WIPP_pnl]
skip_check_regression = True
diff_ascii_output = output_WIPP_pnl-0.pnl

[output_waste_form_wf]
skip_check_regression = True
diff_ascii_output = output_waste_form_wf-0.wf

[output_face_vel-np4]
np = 4
skip_check_regression = True
diff_ascii_output = output_face_vel-np4-qlx-000.tec output_face_vel-np4-qly-000.tec output_face_vel-np4-qlz-000.tec

[output_sc_tecplot]
skip_check_regression = True
diff_ascii_output =  output_sc_tecplot-sec-rank0-obs0-000.tec

[output_coordinates]

[output_aggregate-np2]
np = 2
diff_ascii_output = output_aggregate-np2-obs-1-agg-1.tec

\section{FLUID\_PROPERTY Card}

\begin{frame}[fragile,containsverbatim]\frametitle{FLUID\_PROPERTY}

\begin{itemize}
\item[] \textbf{Purpose:} Define fluid properties
\item[] \textbf{Example uses:}
\begin{itemize}
  \item Assigning a coefficient of diffusion
\end{itemize}
\end{itemize}

\end{frame}

\begin{frame}[fragile]\frametitle{FLUID\_PROPERTY: Examples}

\end{frame}


\end{document}