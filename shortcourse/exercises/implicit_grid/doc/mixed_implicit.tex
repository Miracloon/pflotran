\documentclass{beamer}

\usepackage{comment}
\usepackage{color}
\usepackage{listings}
\usepackage{verbatim}
\usepackage{multicol}
\usepackage{booktabs}
\definecolor{green}{RGB}{0,128,0}

\def\EQ#1\EN{\begin{equation*}#1\end{equation*}}
\def\BA#1\EA{\begin{align*}#1\end{align*}}
\def\BS#1\ES{\begin{split*}#1\end{split*}}
\newcommand{\bc}{\begin{center}}
\newcommand{\ec}{\end{center}}
\newcommand{\eq}{\ =\ }
\newcommand{\degc}{$^\circ$C}

\def\p{\partial}
\def\qbs{\boldsymbol{q}}
\def\Dbs{\boldsymbol{D}}
\def\A{\mathcal A}
\def\gC{\mathcal C}
\def\gD{\mathcal D}
\def\gL{\mathcal L}
\def\M{\mathcal M}
\def\P{\mathcal P}
\def\Q{\mathcal Q}
\def\gR{\mathcal R}
\def\gS{\mathcal S}
\def\X{\mathcal X}
\def\bnabla{\boldsymbol{\nabla}}
\def\bnu{\boldsymbol{\nu}}
\renewcommand{\a}{{\alpha}}
%\renewcommand{\a}{{}}
\newcommand{\s}{{\sigma}}
\newcommand{\bq}{\boldsymbol{q}}
\newcommand{\bz}{\boldsymbol{z}}
\def\bPsi{\boldsymbol{\Psi}}

\def\Li{\textit{L}}
\def\Fb{\textbf{f}}
\def\Jb{\textbf{J}}
\def\cb{\textbf{c}}

\def\Dt{\Delta t}
\def\tpdt{{t + \Delta t}}
\def\bpsi{\boldsymbol{\psi}}
\def\dbpsi{\delta \boldsymbol{\psi}}
\def\bc{\textbf{c}}
\def\dbc{\delta \textbf{c}}
\def\arrows{\rightleftharpoons}

\newcommand{\bGamma}{\boldsymbol{\Gamma}}
\newcommand{\bOmega}{\boldsymbol{\Omega}}
%\newcommand{\bPsi}{\boldsymbol{\Psi}}
%\newcommand{\bpsi}{\boldsymbol{\psi}}
\newcommand{\bO}{\boldsymbol{O}}
%\newcommand{\bnu}{\boldsymbol{\nu}}
\newcommand{\bdS}{\boldsymbol{dS}}
\newcommand{\bg}{\boldsymbol{g}}
\newcommand{\bk}{\boldsymbol{k}}
%\newcommand{\bq}{\boldsymbol{q}}
\newcommand{\br}{\boldsymbol{r}}
\newcommand{\bR}{\boldsymbol{R}}
\newcommand{\bS}{\boldsymbol{S}}
\newcommand{\bu}{\boldsymbol{u}}
\newcommand{\bv}{\boldsymbol{v}}
%\newcommand{\bz}{\boldsymbol{z}}
\newcommand{\pressure}{P}

\newcommand\gehcomment[1]{{{\color{orange} #1}}}
\newcommand\add[1]{{{\color{blue} #1}}}
\newcommand\remove[1]{\sout{{\color{red} #1}}}
\newcommand\codecomment[1]{{{\color{green} #1}}}
\newcommand\redcomment[1]{{{\color{red} #1}}}
\newcommand\bluecomment[1]{{{\color{blue} #1}}}
\newcommand\greencomment[1]{{{\color{green} #1}}}
\newcommand\magentacomment[1]{{{\color{magenta} #1}}}

\begin{comment}
\tiny
\scriptsize
\footnotesize
\small
\normalsize
\large
\Large
\LARGE
\huge
\Huge
\end{comment}

\begin{document}
\title{Mixed Implicit Grid\ldots in a Nutshell}
\author{Emily Stein}
\date{\today}

%\frame{\titlepage}

%-----------------------------------------------------------------------------
\section{Description of Mixed Implicit Grid}

\subsection{Mixed Implicit Grid Conceptual Model}

%\frame{\frametitle{Description of Mixed Implicit Grid}
\begin{frame}\frametitle{Description of Mixed Implicit Grid}
The ``Mixed Implicit Grid Scenario'' demonstrates how to set up a flow and transport problem using an implicit unstructured grid.  The grid includes hexahedral, tetrahedral, wedge, and pyramid cells.  Other features include:
\begin{itemize}
  \item Problem domain: $5 \times 5 \times 5$ m (x $\times$ y $\times$ z)
  \item Grid resolution 15 cells
  \item Flow mode: Richards
  \item Initial conditions: Hydrostatic
  \item Boundary conditions: Recharge and time-varying hydrostatic
  \item Source: Volumetric rate
  \item Maximum time step size: 10 d
  \item Total simulation time: 100 d
\end{itemize}

\end{frame}

%-----------------------------------------------------------------------------
\section{Partial Description of Input Deck: Grid, Regions, TIMESTEPPER}

%-----------------------------------------------------------------------------
\subsection{GRID}

\begin{frame}[fragile]\frametitle{GRID}

\begin{itemize}
  \item Specify grid type
  \item Specify grid file
\end{itemize}


\begin{semiverbatim}

GRID \bluecomment{!unstructured grid implicit}
  TYPE UNSTRUCTURED ./mixed.ugi
END

\end{semiverbatim}

\end{frame}

%-----------------------------------------------------------------------------
\subsection{implicitgrid}

\begin{frame}[fragile]\frametitle{mixedimplicitugi}

\begin{semiverbatim}
15 24 \bluecomment{! 15 cells, 24 vertices}
P 4 5 6 2 1           \bluecomment{! Pyramid}
T 4 3 5 1             \bluecomment{! Tetrahedron}
W 2 7 6 4 9 5         \bluecomment{! Wedge}
... \gehcomment{2 more cells here}
H 19 9 5 12 17 7 6 16 \bluecomment{! Hexahedron}
... \gehcomment{remaining 11 cells here}
5.000000e+00 5.000000e+00 5.000000e+00 \bluecomment{x y z}
5.000000e+00 2.500000e+00 5.000000e+00
5.000000e+00 5.000000e+00 2.500000e+00
5.000000e+00 2.500000e+00 2.500000e+00
2.500000e+00 5.000000e+00 2.500000e+00
... \gehcomment{remaining 19 vertices here}
\end{semiverbatim}

\end{frame}
%-----------------------------------------------------------------------------
\subsection{REGION}

\begin{frame}[fragile,containsverbatim,allowframebreaks]\frametitle{REGION}

\begin{itemize}
  \item Delineate regions in the 3D domain:
  \begin{itemize}
    \item using COORDINATES
    \item using FILE
    \item using COORDINATE
  \end{itemize}
\end{itemize}

\begin{semiverbatim}
REGION all            \bluecomment{! define a region and name it: \greencomment{all}}
  COORDINATES         \bluecomment{! using \redcomment{volume}}
    0.d0 0.d0 0.d0    \bluecomment{! xmin ymin zmin}
    5.d0 5.d0 5.d0    \bluecomment{! xmax ymax zmax}
  /   \bluecomment{! <-- closes out COORDINATES card}
END   \bluecomment{! <-- closes out REGION card}

\newpage
REGION top      \bluecomment{! define region:} \greencomment{top}
  FILE top.ss   \bluecomment{! using a file}
END

REGION west     \bluecomment{! define region:} \greencomment{west}
  FILE west.ss  \bluecomment{! using a file}
END

REGION well  \bluecomment{! define a region for the well}
  COORDINATE 1.25d0 2.91667 1.25d0 \bluecomment{! using \redcomment{point} x y z}
END

REGION middle \bluecomment{! define a region for the obs pt}
  COORDINATE 2.50001d0 2.50001d0 2.50001d0 \bluecomment{! x y z}
END

\end{semiverbatim}

\end{frame}

%-----------------------------------------------------------------------------
\subsection{FILE}

\begin{frame}[fragile]\frametitle{FILE top.ss}

\begin{semiverbatim}
5 \bluecomment{! 5 cell faces}
T 6 2 1     \bluecomment{! Triangular face}
T 2 7 6     \bluecomment{! Vertices listed counterclockwise}
T 8 7 2     \bluecomment{! or clockwise}
Q 17 7 6 16 \bluecomment{! Quadrilateral face}
Q 18 8 7 17
\end{semiverbatim}

\end{frame}

%-----------------------------------------------------------------------------
\subsection{Couplers}

\begin{frame}[fragile]\frametitle{Couplers}

\begin{itemize}
\item Use the same syntax whether REGION is defined with COORDINATE, COORDINATES or FILE.
\end{itemize}

\begin{semiverbatim}

STRATA
  REGION all \bluecomment{! REGION \greencomment{all} is defined with COORDINATES}
  MATERIAL soil1
END

BOUNDARY_CONDITION west
  FLOW_CONDITION initial
  TRANSPORT_CONDITION initial
  REGION west \bluecomment{! REGION \greencomment{west} is defined with FILE}
END

OBSERVATION \bluecomment{! use REGION \greencomment{middle} as an obs pt}
  REGION middle \bluecomment{! REGION \greencomment{middle} is defined with COORDINATE}
END
\end{semiverbatim}

\end{frame}

%-----------------------------------------------------------------------------
\subsection{TIMESTEPPER}

\begin{frame}[fragile]\frametitle{TIMESTEPPER}

\begin{itemize}
\item To check that initial conditions, materials, and grid are good run for only 1 time step
\end{itemize}

\begin{semiverbatim}

TIMESTEPPER FLOW
  MAX_STEPS 1
END
\end{semiverbatim}

\begin{itemize}
\item Skip this block when you are ready to run the full simulation
\end{itemize}

\begin{semiverbatim}

skip \bluecomment{! \greencomment{skip} until encounter \greencomment{noskip}}
TIMESTEPPER FLOW
  MAX_STEPS 1
END
noskip \bluecomment{! begin using input again}
\end{semiverbatim}

\end{frame}

%-----------------------------------------------------------------------------
\end{document}
