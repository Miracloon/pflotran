
\section{Visualization}

Visualization of the results produced by PFLOTRAN can be achieved using several different utilities including commercial and open source software. 
Plotting 2D or 3D output files can be done using the commercial package {\bf Tecplot}, or the opensource packages \href{https://wci.llnl.gov/codes/visit/}{VisIt} and \href{http://www.paraview.org/}{ParaView}. {\bf Paraview} is similar to {\bf VisIt}, and both are capable of remote visualization on parallel architectures.  

Several potentially useful hints on using these packages are provided below.

\subsection{Tecplot}

\begin{itemize}
\item In order to change the default file format in {\bf TecPlot} so that it recognizes {\tt .tec} files, place the following in the tecplot.cfg file:
\end{itemize}

\small
\verb|$!FileConfig FNameFilter {InputDataFile = "*.tec"}|
\normalsize

\subsection{VisIt}

%When using {\bf VisIt} a few useful hints follow:
\begin{itemize}
\item Inactive cells can be omitted by going to {\small\tt Controls -> Subset} and unchecking Material\_ID[0].

\item In 3D to scale the size of one of the coordinate axes go to 
{\tt Controls -> View}
and check box {\small\tt Scale 3D Axes} and set desired scaling factor in box to right.

\item To make 2D plots use {\small\tt Operators -> Slicing -> Slice}.

\item For 3D plots {\small\tt Operators -> Slicing -> ThreeSlice} is useful.

\item Instructions to set up geomechanical and flow data simultaneously in {\bf VisIt}:

%Author: Satish Karra
%Date: 10/08/13
%==========================================================

\begin{enumerate}
\item Open VisIt.
\item Select the $1\times 2$ layout button on Window 1 to open a new window (Window 2).
\item Now make Window 1 active by checking the first button on Window 1.
\item Click on Open and select *-geomech-*.xmf database.
\item Click on Add --> Pseudocolor --> rel\_disp\_z and click Draw (Set Auto apply to avoid clicking Draw repeatedly).
\item Select the domain and rotate it to a preferred angle.
\item Select Window 2 and make it active.
\item Click Open and select pflotran.h5 or pflotran-*.h5 (This file contains all the subsurface flow data).
\item Click on e.g. Add --> Pseudocolor --> Gas Saturation (or other desired variable).
\item Click on Lock view and Lock time on both windows (This will sync views and times on both windows).
\item After a window pops up, select Yes.
\item With Window 2 active select: Operators --> Slicing --> ThreeSlice.
\item Double click on three slice and change x and y to appropriate values.
\item Select Window 1.
\item Select Controls --> Expressions.
\item Click New and add a name (e.g., disp\_vector). Select Vector Mesh Variable.
\item Under Definition, add {<rel\_disp\_x \_lb\_m\_rb\_>,<rel\_disp\_y \_lb\_m\_rb\_>,<rel\_disp\_z \_lb\_m\_rb\_>}, Apply and Dismiss.
\item Click Add --> Vector --> disp\_vector.
\item Double click Vector under pflotran-geomech-*.xmf database:disp\_vector.
\item Select Form and set Scale to e.g. 0.5. Then select Rendering and change magnitude color from Default to difference (select a color scheme of your choice).
\item Apply and Dismiss.
\item Double click Pseudocolor --> rel\_disp\_z. Select e.g. hot\_desaturated for color table. Set opacity to 50\%. Apply and Dismiss.
\item Next, for mesh movement click: Operators --> Transforms --> Displace. After a window pops up, dismiss it.
\item Double click Displace, change Displacement multiplier to e.g. 10000. Click on Displacement Variable --> Vectors --> disp\_vectors. Dismiss the window saying no data etc. Apply and Dismiss.
\item Finally, click the play button to watch movie. Rotate the domain to a convenient angle before doing so.
\end{enumerate}
\end{itemize}

\subsection{Gnuplot, MatPlotLib}

\begin{itemize}

\item For 1D problems or for plotting PFLOTRAN observation, integral flux and mass balance output, the opensource software packages \href{http://www.gnuplot.info/}{gnuplot} and \href{http://matplotlib.org/}{matplotlib} are recommended.
\item With {\bf gnuplot} and {\bf matplotlib} it is possible to plot data from several files in the same plot.
To do this with {\bf gnuplot} it is necessary that the files have the same number of rows, e.g. time history points. The files can be merged during input by using the {\tt paste} command as a pipe: e.g.

\verb|plot '< paste file1.dat file2.dat' using 1:($n1*$n2)|

\noindent
plots the product of variable in file 1 in column {\tt n1} times the variable in file 2 in column {\tt n2} of the merged file.

\item When using {\bf gnuplot} it is possible to number the output file columns with 
{\tt PRINT\_COLUMN\_IDS} added to the {\tt OUTPUT} keyword. This is only useful, however, with {\tt FORMAT TECPLOT POINT} output option.
\end{itemize}
